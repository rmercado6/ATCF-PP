In recent years, a surge in research has been transforming Natural Language Processing (NLP) and Large Language Models
(LLMs).
No longer confined to academia, these technologies are attracting interest due to their potential to revolutionize fields
like healthcare and finance.
Breakthroughs in deep learning architectures and the accessibility of powerful tools and datasets are empowering
researchers to develop LLMs that can not only understand massive amounts of text data but also perform complex tasks
like translation and creative content generation.
This surge in research holds immense promise for the future, with the potential to unlock even more groundbreaking
applications of NLP and LLMs in the years to come.

Domain specific training and fine-tuning of models is an active field of research.
Finance is one of the domains used during training and fine-tuning research.
Recent surveys such as ~\cite{Lee2024} and ~\cite{Li2023} describe the current state of the research on finances in LLMs.
~\cite{Zhao2024} presents an overview of LLMs applications and insights over finances.
The raise of interest over LLMs application in the finance field has driven researchers to build datasets such as
~\cite{Dong2024, El-Haj, Sharma2023}, and even develop models trained on finance specific sata such as BloombergGPT
~\cite{Wu2023}.

The task of text summarization being achieved by LLMs has been actively researched in recent years.
~\cite{Jin2024} provides a survey on text summarization with LLM-based methods.
This proposal focuses on the PEGASUS model ~\cite{Zhang2019} which is a general summarization model trained on news
articles from no specific domain.
Big Bird ~\cite{Zaheer2020} is a summarization centered LLM based on the PEGASUS model.
Summarization with a focus on the Finance domain has been explored in ~\cite{Avramelou2023, passali-etal-2021-towards}.
