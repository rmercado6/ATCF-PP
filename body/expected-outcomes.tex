This project holds the potential to significantly enrich our knowledge in the realm of financial document summarization
using LLMs. The project's contribution hinges on the resulting performance of the fine-tuned model.

In a scenario where the fine-tuned PEGASUS model significantly outperforms the general model on financial document
summarization tasks, the results would be relevant in several ways.
Firstly, it would provide concrete evidence that fine-tuning LLMs with domain-specific data like the FNSPID corpus can
dramatically improve their ability to summarize financial documents.
This would be a major leap forward in the application of NLP for financial purposes.
Secondly, the research would contribute to the development of best practices for financial document summarization using
LLMs.
By analyzing the strengths and weaknesses of the fine-tuned model, researchers could glean valuable insights into the
most effective techniques for tailoring LLMs to capture the nuances of financial language and key metrics.
Finally, the success of the fine-tuned model would suggest that LLMs can effectively capture the relationships between
financial news articles and the information they convey.
This knowledge could be leveraged to develop new methods for analyzing financial news data and identifying trends or
potential risks.

Even if the fine-tuned PEGASUS model performs similarly to or worse than the general model, the research project would
still yield valuable knowledge.
The project's findings could shed light on the potential challenges associated with fine-tuning LLMs for financial
documents.
This might include the need for even larger or more specialized datasets, or the inherent difficulty of capturing the
complexities of financial language within an LLM framework.
The research might also reveal limitations within the FNSPID dataset itself, such as a lack of diversity or insufficient
coverage of specific financial domains.
This knowledge would be crucial for informing future data collection efforts for LLM fine-tuning in finance.
Additionally, the project's findings could suggest that the PEGASUS architecture might not be ideally suited for financial
document summarization.
This would encourage the exploration of alternative LLM architectures or fine-tuning techniques specifically designed for
the financial domain.

In essence, the proposed project has the potential to significantly contribute to our understanding of how LLMs can be
leveraged for financial document summarization, regardless of the outcome.
The findings would be valuable for both researchers and practitioners working in the field of natural language processing
and financial information analysis.
